%----------------------------------------------------------------------------------------
%	PACKAGES AND OTHER DOCUMENT CONFIGURATIONS
%----------------------------------------------------------------------------------------
\documentclass{resume} % Use the custom resume.cls style
\usepackage[left=0.75in,top=0.6in,right=0.75in,bottom=0.6in]{geometry} % Document margins
\usepackage{xeCJK}

\usepackage{bibentry}

\newcounter{qcounter}


%----------------------------------------------------------------------------------------
%	CONTACT SECTION
%----------------------------------------------------------------------------------------
\name{郦乐宁} % Your name
\address{160 Cambridge St, Unit 104, Burlington, MA 01803, USA}
\address{+1~(774)~823~2639 \\ {leningli@outlook.com}
}
\begin{document}

%----------------------------------------------------------------------------------------
%	REFERENCES SECTION
%----------------------------------------------------------------------------------------
\nobibliography{refs}
\bibliographystyle{ieeetr}

%----------------------------------------------------------------------------------------
%	RESEARCH INTERESTS SECTION
%----------------------------------------------------------------------------------------
\begin{rSection}{研究兴趣}
    强化学习 \, $\circ$ \, 优化控制 \, $\circ$ \, 博弈论 \, $\circ$ \, 形式化方法
\end{rSection}


%----------------------------------------------------------------------------------------
%	EDUCATION SECTION
%----------------------------------------------------------------------------------------
\begin{rSection}{教育背景}
\begin{hSubsection}
{机器人工程博士}{2016.8 - 至今} {\it 伍斯特理工学院} {伍斯特, 马萨诸塞, 美国}{成绩: \bf{4.0}/4.0}
\end{hSubsection}

\begin{hSubsection}
{机器人工程硕士}
{2016.8 - 2021.5}
{伍斯特理工学院}
{伍斯特, 马萨诸塞, 美国} 
{成绩: \bf{3.70}/4.0}
\end{hSubsection}

\begin{hSubsection}
{计算机科学硕士}
{2014.8 - 2016.5}
{伍斯特理工学院}
{伍斯特, 马萨诸塞, 美国}
{成绩: \bf{3.64}/4.0}
\end{hSubsection}

\begin{hSubsection}
{信息安全和英语语言与文学学士}{2010.9 - 2014.7}
{\it 哈尔滨工业大学(威海)}{威海, 山东, 中国} 
{成绩: 85/100} ({\bf 前 10\%})
\end{hSubsection}
\end{rSection}

\begin{rSection}{资质证书}
\begin{hSubsection}
{大学教学认证}
{2017.6 - 2019.8}
{Higher Education Consortium of Central Massachusetts (HECCMA)}
{伍斯特, 马萨诸塞, 美国}
{培训高等教育教学方法}
\end{hSubsection}

\begin{hSubsection}
{网站开发基础认证}
{2015.6 - 2015.8}
{哈佛大学}
{剑桥, 马萨诸塞, 美国}
{学习开发网站的方法, 包括CSS, HTML, JavaScript}
\end{hSubsection}
\end{rSection}

%----------------------------------------------------------------------------------------
%	REPRESENTATIVE PROJECTS SECTION
%----------------------------------------------------------------------------------------
\begin{rSection}{代表项目}
    \begin{rSubsection}{DARPA 机器人大赛}{2014.8 - 2015.5}{研究员}{伍斯特, 马萨诸塞, 美国}
        \item 与卡内基梅隆大学合作研究人形机器人Atlas
        \item 开发了运动规划理论用于机器手的操控
    \end{rSubsection}
    \begin{rSubsection}{DARPA Serial Interactions in Imperfect Information Games Applied to Complex Military Decision Making (SI3-CMD)}{2019.1 - 2020.8}{研究员}{伍斯特, 马萨诸塞, 美国}
        \item 与 Scientific Systems Company Inc (SSCI) 合作开发欺骗规划博弈学理论
        \item 提高了在不对称信息下的目标成功率
    \end{rSubsection}
    \begin{rSubsection}{在形式化规约下的随机系统的优化控制与强化学习}{2016.8 - 2022.8}{研究员}{伍斯特, 马萨诸塞, 美国}
        \item 开发了一种无模型的强化学习方法在形式化规约下用于随机规划
        \item 解决了在强化学习中稀疏奖励的问题
\end{rSubsection} 
\end{rSection}

%----------------------------------------------------------------------------------------
%	INDUSTRY EXPERIENCE SECTION
%----------------------------------------------------------------------------------------
\begin{rSection}{职业经历}
\begin{rSubsection}{高级软件工程师}{2021.10 - 2022.8}{Berkshire Grey}{贝德福德, 马萨诸塞, 美国}
    \item 开发算法抓取不明属性的物体
\end{rSubsection}

\begin{rSubsection}{软件工程师实习}{2015.6 - 2016.1}{Rudolph Technologies}{图克斯伯里, 马萨诸塞, 美国}
\item 开发自动化工具实现代码迁移
\item 开发代码收集晶片数据
\item 分析数据开发代码帮助客户提升晶圆产量
\end{rSubsection}

\begin{rSubsection}{软件工程师实习}{2013.7 - 2013.8}{东软集团}{大连, 辽宁, 中国}
\item 开发了高效的地图管理系统 
\end{rSubsection}
\end{rSection}

%----------------------------------------------------------------------------------------
%	TEACHING EXPERIENCE SECTION
%----------------------------------------------------------------------------------------
\begin{rSection}{教学经历}
\begin{rSubsection}
{助教}{2022.8 - 2022.12}{RBE 549. 计算机视觉, 伍斯特理工学院}{伍斯特, 马萨诸塞, 美国}
\item 帮助设计新的计算机视觉课程
\item 课程网站:\url{https://nitinjsanket.github.io/teaching/rbe549/fall2022.html}
\item 答疑和批改作业
\end{rSubsection}

\begin{rSubsection}{助教}{2018.8 - 2018.12}{RBE 549. 计算机视觉, 伍斯特理工学院}{伍斯特, 马萨诸塞, 美国}
\item 负责图像过滤的教学任务
\item 答疑和批改作业
\end{rSubsection}

\begin{rSubsection}
{助教}{2020.8 - 2021.5}{RBE 3001. \& 3002. 统一机器人技术 III \& IV, 伍斯特理工学院}{伍斯特, 马萨诸塞, 美国}
\item 带领3D打印机器手和移动机器人的实验
\item 答疑, 设计和批改实验报告、课程项目、作业
\end{rSubsection}

\begin{rSubsection}{助教}{2017.8 - 2018.5}{RBE 1001. 机器人学导论, 伍斯特理工学院}{伍斯特, 马萨诸塞, 美国}
\item 日常管理5个本科生助教
\item 答疑, 设计和批改实验报告、课程项目、作业
\end{rSubsection}
\end{rSection}




%----------------------------------------------------------------------------------------
%	SKILLS SECTION
%----------------------------------------------------------------------------------------
\begin{rSection}{计算机技能}
    \begin{tabular}{ @{} >{\bfseries}l @{\hspace{6ex}} l }
    编程语言 & C/C++, Python, LaTex, MATLAB \\
    机器人工具 & Ubuntu, ROS, RVIZ, Gazebo, OpenRave \\
    软件 & PyTorch, TensorFlow, Git, Elasticsearch, Kubernetes, QtCreator
    \end{tabular}
\end{rSection}

%----------------------------------------------------------------------------------------
%	AWARDS SECTION
%----------------------------------------------------------------------------------------
\begin{rSection}{过往荣誉}
\begin{nSubsection}
{优秀毕业生}
{2014.7}
{哈尔滨工业大学(威海)}{威海,山东,中国}
\end{nSubsection}

\begin{nSubsection}
{Alex F. Backlin 奖学金, \$1610} {2021.1}
{伍斯特理工学院}
{伍斯特, 马萨诸塞, 美国}
\end{nSubsection}

\begin{nSubsection}
{旅行奖励, \$250}{2019.10}
{理海大学}{伯利恒, 宾夕法尼亚, 美国}
\end{nSubsection}

\begin{nSubsection}
{研究生旅行奖励, \$500}{2019.10}
{伍斯特理工学院}{伍斯特, 马萨诸塞, 美国}
\end{nSubsection}

\begin{nSubsection}
{研究生旅行奖励, \$400}{2019.3}
{伍斯特理工学院}{伍斯特, 马萨诸塞, 美国}
\end{nSubsection}

\begin{nSubsection}
{研究生旅行奖励, \$1000}{2017.6}
{伍斯特理工学院}{伍斯特, 马萨诸塞, 美国}
\end{nSubsection}
\end{rSection}


%----------------------------------------------------------------------------------------
%	EXTRACURRICULAR ACTIVITIES SECTION
%----------------------------------------------------------------------------------------
\begin{rSection}{课外经历}
    \begin{rSubsection}{主席}{2019.1 - 2020.5}{研究生学生会}{伍斯特, 马萨诸塞, 美国}
        \item 管理研究生学生会
        \item 代表研究生学生会与学校行政部门合作
        \item 在研究生住宿问题上与研究生办公室合作向校董会报告演讲
        \item 代表研究生学生会帮助寻找教务处处长
    \end{rSubsection}
    \begin{rSubsection}{志愿者}{2013.7 - 2013.9}{拉萨儿童福利院}{拉萨, 西藏, 中国}
        \item 为贫困儿童募款
        \item 为学习困难的学生讲授课程内容
\end{rSubsection}
\end{rSection}

%----------------------------------------------------------------------------------------
%	ORGANIZATION MEMBERSHIPS SECTION
%----------------------------------------------------------------------------------------
\begin{lSection}{会籍}
    \item {\bf 会员}, IEEE
    \item {\bf 会员}, IEEE Young Professionals
    \item {\bf 会员}, IEEE Robotics and Automation Society
    \item {\bf 会员}, Association for Women in Mathematics
    \item {\bf 会员}, Alpha Chapter of Rho Beta Epsilon (荣誉团体), 伍斯特理工学院.
\end{lSection}

%----------------------------------------------------------------------------------------
%	SOCIAL MEDIA SECTION
%----------------------------------------------------------------------------------------
\begin{lSection}{社交}
    \item 领英: \url{https://www.linkedin.com/in/lening-li}
    \item 个人: \url{https://lening.li}
    \item GitHub: \url{https://github.com/leelening}
    \item Google 学术: \url{https://scholar.google.com/citations?user=KWUJ10wAAAAJ\&hl=en}
\end{lSection}

%----------------------------------------------------------------------------------------
%	LANGUAGE PROFICIENCY SECTION
%----------------------------------------------------------------------------------------
\begin{lSection}{语言}
    \item {\bf 中文}, 母语
    \item {\bf 英语}, 精通
\end{lSection}

%----------------------------------------------------------------------------------------
%	EDITORIAL ACTIVITIES SECTION
%----------------------------------------------------------------------------------------
\begin{rSection}{审稿}
    \begin{lSubsection}{期刊审稿}
        \item IEEE Control Systems Letters (L-CSS)
        \item IET Cyber-Systems and Robotics
    \end{lSubsection}
    \begin{lSubsection}{会议审稿}
        \item International Conference on Robotics and Automation (ICRA) 2017 2021.
        \item American Control Conference (ACC), 2019 2021
        \item IEEE Conference on Decision and Control (CDC), 2018 2019 2020
        \item IEEE/RSJ International Conference on Intelligent Robots and Systems (IROS), 2020
        \item International Conference on Ubiquitous Robots (UR), 2020
    \end{lSubsection}
\end{rSection}

%----------------------------------------------------------------------------------------
%	PUBLICATIONS SECTION
%----------------------------------------------------------------------------------------
\begin{rSection}{论文}
    \begin{pSubsection}{准备中}{U}
        \item L.Li and J. Fu, "Topological Order Guided Actor-Critic Modular Learning of Continuous Systems with Temporal Objectives ", 2022
    \end{pSubsection}

    \begin{pSubsection}{会议}{C}
        \item \bibentry{li2019topological}
        \item \bibentry{li2019approximate}
        \item \bibentry{li2017sampling}
        \item \bibentry{li2016birrtopt}
        \item \bibentry{atkeson2015no}
        \item \bibentry{li2021policy}
    \end{pSubsection}

    \begin{pSubsection}{期刊}{J}
        \item \bibentry{chen2018building}
        \item \bibentry{dedonato2017team}
        \item \bibentry{atkeson2016happened}
        \item \bibentry{li2020dynamic}
    \end{pSubsection}
    
    \begin{pSubsection}{书籍}{B}
        \item \bibentry{atkeson2018achieving}
        \item \bibentry{atkeson2018happened}
    \end{pSubsection}

    \begin{pSubsection}{论文}{T}
        \item \bibentry{limaster}
        \item \bibentry{libs}
        \item \bibentry{liba}
        \item L.Li, {\it Optimal Control and Reinforcement Learning for Stochastic Systems under Temporal Logic Specifications.} Doctoral  thesis, Worcester Polytechnic Institute, 2022
    \end{pSubsection}
    
    \begin{pSubsection}{演讲}{P}
        \item {Topological approximate dynamic programming under temporal logic constraints.} Poster Presentation, Northeast Robotics Colloquium (NERC), 2019.
        \item {Topological approximate dynamic programming under temporal logic constraints.} Poster Presentation, Robot Learning Workshop, 2019.
        \item {Approximate dynamic programming with probabilistic temporal logic constraints}, Paper Presentation, American Control Conference (ACC), 2019.
        \item {Sampling-based approximate optimal temporal logic planning}, Paper Presentation, IEEE International Conference on Robotics and Automation (ICRA), 2017
        \item {Birrtopt: A combined sampling and optimizing motion planner for humanoid robots} Poster Presentation, International Conference on Humanoid Robots (Humanoids), 2016.
        \item {Birrtopt: A combined sampling and optimizing motion planner for humanoid robots} Paper Presentation, Northeast Robotics Colloquium (NERC), 2016.
    \end{pSubsection}
\end{rSection}



\end{document}
