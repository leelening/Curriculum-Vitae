\documentclass{resume}
\usepackage[left=0.75in,top=0.6in,right=0.75in,bottom=0.6in]{geometry}

% Chinese support (Overleaf-safe)
\usepackage[UTF8]{ctex}

\usepackage{bibentry}
\usepackage{url}
\newcounter{qcounter}

%----------------------------------------------------------------------------------------
% 联系方式
%----------------------------------------------------------------------------------------
\name{郦乐宁}
\address{Wayland, MA 01778, 美国}
\address{+1 (774) 823 2639 \\ leningli@outlook.com}

\begin{document}

\nobibliography{refs}
\bibliographystyle{ieeetr}

%========================================================================================
% 研究兴趣
%========================================================================================
\begin{rSection}{研究兴趣}
强化学习;最优控制;博弈论;形式化方法
\end{rSection}

%========================================================================================
% 教育背景
%========================================================================================
\begin{rSection}{教育背景}

{\bf 伍斯特理工学院(Worcester Polytechnic Institute, WPI)} \hfill 美国 马萨诸塞州 伍斯特 \\
{\bf 机器人工程博士(Ph.D.)} \hfill 2016.08 -- 2022.12
\begin{list}{$\circ$}{\leftmargin=1.5em}
\item 博士论文:\textit{基于时间逻辑约束的随机系统最优控制与强化学习}
\end{list}

{\bf 机器人工程硕士(M.S.)} \hfill 2016.08 -- 2018.05 \\

{\bf 计算机科学硕士(M.S.)} \hfill 2014.08 -- 2016.05
\begin{list}{$\circ$}{\leftmargin=1.5em}
\item 硕士论文:\textit{Birrtopt:面向 Atlas 机器人的运动规划统一软件框架}
\end{list}

{\bf 哈尔滨工业大学(Harbin Institute of Technology, HIT)} \hfill 中国 \\
{\bf 计算机科学学士(B.S.)} \hfill 2010.09 -- 2014.07
\begin{list}{$\circ$}{\leftmargin=1.5em}
\item 荣誉:Summa Cum Laude(专业前 5\%)
\item 本科论文:\textit{基于 Contourlet 变换的图像压缩}
\end{list}

{\bf 英语语言与文学学士(B.A.)} \hfill 2011.09 -- 2014.07
\begin{list}{$\circ$}{\leftmargin=1.5em}
\item 学位论文:\textit{《〈恋爱中的女人〉中的男性沙文主义研究》}
\end{list}

\end{rSection}

%========================================================================================
% 资格认证
%========================================================================================
\begin{rSection}{资格认证}

\begin{hSubsection}
{REC 基金会认证教练}
{2023.09 -- 2024.09}
{Robotics Education \& Competition Foundation}
{线上}
{指导与培养机器人竞赛团队,促进学生在 STEM 与工程实践方面的发展。}
\end{hSubsection}

\begin{hSubsection}
{FIRST Tech Challenge(FTC)认证教练}
{2022.11 -- 2023.05}
{FIRST Robotics}
{线上}
{指导 FTC 机器人团队进行机器人设计、编程与竞赛策略制定。}
\end{hSubsection}

\begin{hSubsection}
{大学教学资格认证}
{2017.06 -- 2019.08}
{马萨诸塞州中部高等教育联盟(HECCMA)}
{美国 伍斯特}
{接受循证教学法系统培训,具备高校课程设计与授课能力。}
\end{hSubsection}

\end{rSection}

%========================================================================================
% 行业与学术顾问经历
%========================================================================================
\begin{rSection}{行业与学术顾问经历}

\begin{rSubsection}{机器人实验室顾问}{2025.12 -- 至今}{哈佛大学}{美国 马萨诸塞州 剑桥}
\item 指导机器人方向学生科研项目,重点关注算法设计、实验严谨性与结果可复现性。
\item 提供涵盖强化学习、规划与控制以及安全导向系统设计的技术指导,连接学术研究与已部署机器人系统。
\item 通过研究评审、项目范围界定与结构化反馈,提升实验室研究效率与整体质量。
\end{rSubsection}

\begin{rSubsection}{高级软件工程师}{2022.10 -- 至今}{Symbotic}{美国 马萨诸塞州 威明顿}
\item 设计并部署用于大规模机器人仓储系统的可扩展多智能体(multi-agent)路径规划与协同算法(C++)。
\item 通过改进控制设计、状态估计与容错行为架构,提升机器人集群层面的系统鲁棒性。
\item 优化实时决策流水线,支持成千上万台自主机器人在严格时延约束下稳定运行。
\end{rSubsection}

\begin{rSubsection}{高级软件工程师}{2021.10 -- 2022.08}{Berkshire Grey}{美国 马萨诸塞州 贝德福德}
\item 主导感知与操作算法开发,用于未知 SKU 的自动抓取(ROS、C++、Python)。
\item 通过重构进程间通信与执行流水线,显著降低端到端系统延迟。
\end{rSubsection}

\begin{rSubsection}{软件工程师实习}{2015.06 -- 2016.01}{Rudolph Technologies}{美国 马萨诸塞州 特克斯伯里}
\item 设计自动化工具,将多套遗留代码库迁移至统一软件平台。
\item 开发方法以提升晶圆缺陷数据采集与分析流程的准确性。
\end{rSubsection}

\begin{rSubsection}{软件工程师实习}{2013.07 -- 2013.08}{东软集团(Neusoft)}{中国}
\item 构建地图管理系统,支持高效的插入、删除与编辑操作。
\end{rSubsection}

\end{rSection}

%========================================================================================
% 代表性技能
%========================================================================================
\begin{rSection}{代表性技能}
\begin{tabular}{ @{} >{\bfseries}l @{\hspace{6ex}} l }
编程语言: & C/C++, Python, MATLAB \\
机器人与控制: & 运动规划,最优控制,强化学习 \\
机器人系统: & ROS,ROS 2,分布式机器人系统 \\
机器学习: & PyTorch,TensorFlow \\
语言能力: & 英语(流利),中文(母语)
\end{tabular}
\end{rSection}

%========================================================================================
% 教学经历
%========================================================================================
\begin{rSection}{教学经历}

\begin{rSubsection}{助教}{2022.08 -- 2022.12}{RBE 549 计算机视觉}{伍斯特理工学院}
\item 参与新计算机视觉课程的设计与授课。
\item 讲授课程内容,并通过结构化答疑支持学生学习。
\item 课程网站:https://nitinjsanket.github.io/teaching/rbe549/fall2022.html
\end{rSubsection}

\begin{rSubsection}{助教}{2020.08 -- 2021.05}{RBE 3001 \& 3002 统一机器人系统 III \& IV}{伍斯特理工学院}
\item 指导 3D 打印机械臂控制与移动机器人导航实验。
\item 设计并评估课程项目、实验报告与作业。
\end{rSubsection}

\begin{rSubsection}{助教}{2018.08 -- 2018.12}{RBE 549 计算机视觉}{伍斯特理工学院}
\item 讲授课程内容并主持答疑时间。
\end{rSubsection}

\begin{rSubsection}{助教}{2017.08 -- 2018.05}{RBE 1001 机器人导论}{伍斯特理工学院}
\item 管理并指导 5 名本科生助教。
\item 设计并评估课程项目、实验报告与作业。
\end{rSubsection}

\end{rSection}

%========================================================================================
% 代表性研究项目
%========================================================================================
\begin{rSection}{代表性研究项目}

\begin{rSubsection}{DARPA Robotics Challenge}{2014.08 -- 2015.05}{研究人员}{美国}
\item 与卡内基梅隆大学团队合作,参与 Boston Dynamics Atlas 人形机器人项目。
\item 设计用于开门、阀门操作与工具抓取的机械臂运动规划算法。
\item 主导人机交互界面设计,团队在 24 支参赛队伍中排名第 7。
\end{rSubsection}

\begin{rSubsection}
{DARPA SI3-CMD:复杂军事决策中的不完美信息序列交互}
{2019.01 -- 2020.08}{研究人员}{美国}
\item 与 Scientific Systems Company Inc.(SSCI)合作,开发基于博弈论的欺骗性规划框架与 Python 软件。
\item 通过利用信息不对称与策略性欺骗,提高任务目标达成概率。
\item 提出面向时间逻辑目标的动态超博弈(hypergame)解概念。
\end{rSubsection}

\begin{rSubsection}
{基于时间逻辑约束的随机系统最优控制与强化学习}
{2016.08 -- 2022.08}{研究人员}{美国}
\item 构建将概率时间逻辑规范转化为具满足性保证的机会约束控制问题的理论框架。
\item 提出适用于连续随机系统的可扩展无模型强化学习方法,提高样本效率。
\end{rSubsection}

\end{rSection}

%========================================================================================
% 荣誉与奖励
%========================================================================================
\begin{rSection}{荣誉与奖励}

\begin{nSubsection}
{Alex F. Backlin 基金奖学金}{2021.01}
{伍斯特理工学院}{美国}
\end{nSubsection}

\begin{nSubsection}
{研究生差旅奖}{2017.06}
{伍斯特理工学院}{美国}
\end{nSubsection}

\begin{nSubsection}
{研究生差旅奖}{2019.03}
{伍斯特理工学院}{美国}
\end{nSubsection}

\begin{nSubsection}
{研究生差旅奖}{2019.10}
{伍斯特理工学院}{美国}
\end{nSubsection}

\begin{nSubsection}
{差旅资助奖}{2019.10}
{利哈伊大学}{美国}
\end{nSubsection}

\end{rSection}

%========================================================================================
% 课外活动与领导力
%========================================================================================
\begin{rSection}{课外活动与领导力}

\begin{rSubsection}{研究生学生会主席}{2019.01 -- 2020.05}{伍斯特理工学院}{美国}
\item 负责研究生学生会整体治理与运作。
\item 代表研究生群体与学校管理层进行日常沟通。
\item 与研究生院合作,将研究生住房问题提交董事会讨论。
\item 作为成员参与校长遴选委员会并提出建议。
\end{rSubsection}

\begin{rSubsection}{志愿者}{2013.07 -- 2013.09}{拉萨儿童福利院}{中国}
\item 为低收入家庭儿童筹集教育经费。
\item 为儿童提供语文、数学与英语辅导。
\end{rSubsection}

\end{rSection}

%========================================================================================
% 学术与专业组织成员
%========================================================================================
\begin{rSection}{学术与专业组织成员}
\item IEEE 会员
\item IEEE Young Professionals 会员
\item IEEE Robotics and Automation Society 会员
\item Association for Women in Mathematics 会员
\item Rho Beta Epsilon 荣誉学会(WPI Alpha Chapter)
\end{rSection}

%========================================================================================
% 学术服务
%========================================================================================
\begin{rSection}{学术服务}

\begin{lSubsection}{期刊审稿人}
\item IEEE Robotics and Automation Letters (RA-L)
\item IET Cyber-Systems and Robotics
\item IEEE Transactions on Intelligent Transportation Systems
\item Discover Robotics
\item The Journal of Supercomputing
\end{lSubsection}

\begin{lSubsection}{会议审稿人}
\item International Conference on Robotics and Automation (ICRA)
\item American Control Conference (ACC)
\item IEEE Conference on Decision and Control (CDC)
\item IEEE/RSJ International Conference on Intelligent Robots and Systems (IROS)
\item International Conference on Ubiquitous Robots (UR)
\item European Control Conference (ECC)
\end{lSubsection}

\end{rSection}

%========================================================================================
% 学术主页与社交媒体
%========================================================================================
\begin{rSection}{学术主页与社交媒体}
LinkedIn: \url{https://www.linkedin.com/in/lening-li/} \\
个人主页: \url{https://lening.li} \\
GitHub: \url{https://github.com/leelening} \\
Google Scholar: \url{https://scholar.google.com/citations?user=KWUJ10wAAAAJ}
\end{rSection}

%========================================================================================
% 论文发表(保持英文)
%========================================================================================
\begin{rSection}{论文发表}

\begin{pSubsection}{Manuscripts in Preparation}{U.}
\item \bibentry{li2023topological}
\item \bibentry{li2021policy}
\end{pSubsection}

\begin{pSubsection}{Conferences}{C.}
\item \bibentry{atkeson2015no}
\item \bibentry{li2016birrtopt}
\item \bibentry{li2017sampling}
\item \bibentry{li2019topological}
\item \bibentry{li2019approximate}
\item \bibentry{li2023synthesis}
\item \bibentry{li2023probabilistic}{(录用率:15\%)}
\end{pSubsection}

\begin{pSubsection}{Journals}{J.}
\item \bibentry{dedonato2017team}
\item \bibentry{chen2018building}
\item \bibentry{li2022dynamic}
\end{pSubsection}

\begin{pSubsection}{Chapters}{Ch.}
\item \bibentry{atkeson2018achieving}
\item \bibentry{atkeson2018happened}
\end{pSubsection}

\end{rSection}

\end{document}
